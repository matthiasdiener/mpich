\documentstyle[11pt,epsf,handpage]{article}
\pagestyle{empty}
\begin{document}
\pagestyle{empty}
\title{PORTS: POrtable RunTime Systems Consortium}

\date{}
\maketitle

\section*{Description}

PORTS is a consortium of research universities, national laboratories,
and computer vendors interested in developing standard runtime support
for heterogeneous, multithreaded language compilers.  Specific goals
of the group are to:

\begin{enumerate}
\item
Share experience in the development of runtime support for task-
and data-parallel languages.
\item
Identify opportunities for code sharing among various groups
working in this area.
\item
Design and implement a common runtme system, with the goal of
enhancing code reuse and interoperability. 
\end{enumerate}

The PORTS group has met four times since Supercomputing 94.  A first
product of the consortium was the definition of PORTS0, a standard
lightweight thread and reentrant function interface.  An
implementation of this interface has been constructed at Argonne
National Laboratory and the California Institute of Technology,
building on software developed in the Nexus project.


\section*{Documentation}

Information about PORTS is available at the following URL:
\begin{center}
{\tt http://www.cs.uoregon.edu:80/paracomp/ports/}
\end{center}
A portable implementation of the PORTS0 library is available via
ftp:
\begin{center}
{\tt ftp://www.mcs.anl.gov/pub/ports/}
\end{center}
Additional information about the PORTS0 interface (and Nexus) is available
at the URL:
\begin{center}
{\tt http://www.mcs.anl.gov/nexus/}
\end{center}

\contact{Ian Foster}
\email{foster@mcs.anl.gov}
\phone{(708) 252-4619}
\makeinfo

\end{document}


The first action of the group was to propose a charter for PORTS.
The following three items summarizes the goal of the endeavor
\begin{itemize}

\item To design and build inter-operable task parallel programming systems.
Initially, this effort will be organized as a study group that
will focus on the specification of a common runtime system
for task parallelism.  Such a runtime system would be at the level
of a compiler target.

\item The direction of this effort 
includes the integration of task and data parallel programming. It is
not the intention of this group to focus on real-time, embedded systems
or fault tolerance.

\item To Identify opportunities for code sharing among projects.

\end{itemize}
The construction of a common runtime system to be used as a compiler
target for task parallel programming is one example of code sharing.
However, it was observed that the full runtime environment of
any programming language consists of several levels. The lowest
of these provides the basic mechasims in which programs interact
with the hardware and operating system.  Higher levels of the
runtime system implement increasingly higher levels of the programming language
semantics.  It may be the case that languages that share
certain semantic ideas will be able to exploit common runtime structures at 
higher levels.  While there was some feeling that the a full
multi-layered approach to the runtime system design was desirable,
it was decided that the group should focus first on the foundation
layer for task parallel computation.

The remainder of the meeting focused on three major items.
First was a discussion of the status of several current projects.
That was followed by a lengthy discussions of the desired attributes
of the basic runtime layer.  The end of the day was left to formulate
action items.

\section{Project Reports}

Dennis Gannon gave a brief report on the pC++ project.  This
system currently generates SPMD code for the Intel Paragon, CM-5,
BBN TC2000, KSR-1, Sequent and SP-1 and PVM networks of workstations.
The current runtime system is single threaded in each CPU, but it now supports
remote service requests.  A new version of the language and compiler
is under design that will be based on active (threaded) global objects.
Compatibility with CC++, Fortran-M, HPF and
(the new) multi-threaded Vienna Fortran is considered a major
goal.  Consequently, the outcome of the Ports project is very
important to this group.

Carl Kesselman reported that Nexus, the runtime system for both
CC++ and Fortran-M is now working in a number of environments.
These include Solaris threads linking sparc-10 systems over tcp/ip,
Sun-OS, IBM RS6000 systems and DCE, Paragon OSF/1 pthreads.
All of these systems work in a fully pre-emptive environment.
Communication is over local memory and tcp/ip.  Carl also reported
that IBM is considering modifications to the SP-2 threads system
to allow remote service requests.  Initial timing results show
that Nexus, running in this pre-emptive mode, can outperform
a popular message passing library.
Carl also reported that the alpha release of CC++ is now
available and Fortran-M will soon be ported to Nexus.

Mathew Haines described Chant, a runtime layer being built at
ICASE to support research with a task parallel HPF and Vienna
Fortran.  Chant is designed for machines with fast communication.
The goal is to have threads do message passing rather than processors
and to do this without excessive buffer copy operations.
The approach is to extend the pthreads interface to support MPI.
Viewed as layers, the foundation of Chant is light weight thread
system and a communications library.  The next layer is
a point-to-point message service between threads.  Built on
top of this is a remote service request layers and global thread
operations.  

Another ICASE project, Macrame, defines an architectural model and
an object semantics for threads. Macrame defines a semantic
layer between Chant and the programming language.
It contains the concept of a Ropes, which represents
a set of threads with enough common identity that they can carry
out collective operations.

Rajive Bagrodia described the discrete event simulation work
being done at UCLA.  This project will also be based on Nexus.
Rajive brings lots of experience in the areas of scheduling 
(such as termination detection) and
dynamic load balancing to the PORTS group. 
One of the areas he has worked on is a dynamic system to explore
task suspend-and-wake-up behaviors.  In addition, he has worked
on the parallel language UC which started as a data parallel system
but which has also integrated task parallelism.
        
\section{Requirements}

The next task of the group was to discuss a set of requirements
for the Ports project.  Fourteen different topics were covered
in the discussion and are summarized below.

\begin{enumerate}
\item
{\em Target Machines}.  It is the desire of the group to target environments
where a task parallel model of computation makes sense.  This
includes large MIMD multiprocessors  with either shared or distributed
memory semantics and networks of servers and workstations.
This does not include single SIMD systems, but it is important
to note that a SIMD task may be an important component of a task
parallel computation distributed over a network which includes
such a machine.

\item
{\em Heterogeneity.}  The demand for programming system that exploit
heterogeneous networked computing environments is growing.  In particular,
applications that will run across the NII must incorporate support
for heterogeneity.  It was noted that this issues
is distinct from that of portability and it has strong implications
for topics like thread migration.
The questions to be considered are:
\begin{itemize}
\item How much do we need to give up in performance?
\item How much are we prepared to give up?
\item When can the compiler do optimization for a special features
of given hardware platform.
\end{itemize}
It was noted that compiler options may be used to build a version
of a program that assumes the code will be run on a homogeneous 
multicomputer or network.  In this case, a leaner, more specialized
runtime library
is possible.  The disadvantage with this solution is that one
must maintain multiple runtime libraries and objects linked against
them will not be able to interact.

\item
{\em Inter-operability.}  Modern programming systems must allow
applications to be built from a mix of programming styles and languages.
This means a program written in one language should be able to invoke 
functions written in another language.  If two languages have a 
way to describe the same object, then there should be some way for
them to share it. 
For traditional sequential machines it was sufficient to make sure
that simple subroutine calling and stack handling conventions were
observed. More complex problems arise when languages need to share
heap space or garbage collection facilities.  In the case of parallel
and distributed systems, the runtime layer described here becomes
the critical link.  As described above, the runtime may be partitioned
into layers.  Clearly a minimum requirement for inter-operability
between two languages is the basic thread and communication mechanisms
that are fundamental to the Ports model.  However, an open
question for the PORTS group is how many of the higher level semantic
layers of the runtime system may be shared.

An alternative is to follow the lead of the OMG and define a interface
definition language and some form of object broker mechanism that
will allow system to interact.

\item
{\em Performance Analysis and Monitoring.}
It is essential to incorporate some interface to performance analysis
and measurement within the design of the runtime system.  The
basic requirements include common library timers and event logging
mechanism.  Clearly, it would be advantageous to have a mechanism
to measure the life time and activities of a thread, but in a preemptive
environment this is very difficult without adding  additional
state and increased context switching time to the thread.

In addition to thread life time measurement, another important
performance evaluation hook is memory hierarchy and message
traffic behavior.  For example, if the runtime system provides
a virtual shared memory or global name space, it is important to
be able to record the cost of remote references versus local
ones.


\item
{\em Debugging.} No programming system is complete with out
a debugging tools.  The Ports group feels strongly that vendor
supported tools must be used whenever possible.  In general
the Ports group found the state-of-the-art in parallel debugging
to be very depressing.  The group will support any and all
efforts to build good extensible parallel debuggers.

\item
{\em Scheduling.}  The issue of scheduling is critical to
the design of a thread based or task parallel runtime system.
In the case of priority based scheduling it is important to
be able to handle very high priority operations like
remote service requests.  We also we may need ways
to write custom schedulers.  For example, the a scheduler
that can evaluate conditional expressions that determine
the need to schedule another task without requiring a context
switch to that task.  ``Gang scheduling'' is important for 
parallel execution of ropes of threads (see collective operations
below).  It is imperativeimperative that the runtime system avoid 
busy-waiting.  One of the important uses of threads in
a task parallel programming system is to hide message latency.
Consequently, efficient scheduling is critical to overall performance.
 without requiring a context
switch to that task.  ``Gang scheduling'' is important for 
parallel execution of ropes of threads (see collective operations
below).  It is imperative that the runtime system avoid 
busy-waiting.  One of the important uses of threads in
a task parallel programming system is to hide message latency.
Consequently, efficient scheduling is critical to overall performance.

\item
{\em Thread Functionality.}  The general opinion of the Ports group
is that the thread functionality in the base runtime system layer
should be a major subset of  posix threads.  In particular, 
key attributes of threads include:stack size, scheduling policy, 
and scheduling policy parameters, management of threads 
should support  operators like crate, join, delete, equal.
Threads should be able to have local data supported by operators
like key-create, etc.  Synchronization should be supported with
at least mutex and condition variables.


\item
{\em Collective Operations.}  For large scale parallelism, it is
impractical to schedule threads sequentially for the concurrent
evaluation of some part of a program.  Consequently, Ports will
need some form of thread collective.  A rope is a set of threads
that can be created and each assigned a task (such as a loop iterate)
with minimal overhead.  Threads within a rope should be able
to identify the other threads in the rope and there should be
synchronization mechanisms that allow collective operations like
barriers, scans, broadcasts and reductions.  
It is important that the runtime system
level rope structure provide mechanism and not policy.  Different
language may use ropes in very different ways.

\item
{\em Node Abstractions.} In order to facilitate any discussion
of remote versus local computation on parallel or distributed
systems, it is very helpful to have a common language to describe
the attributes of the hardware and OS of these machines.
The following terms will be used in Ports documents.
\begin{itemize} 
\item A {\em context} is an address space in which a computation
may operate. 
\item A {\em node} is the smallest unit of hardware upon which a context can
be assigned. A node consists of a computing engine (one or more CPUs)
and a shared memory system.  Multiple contexts may be assigned
to a given node and CPUs may be shared between the contexts on a given
node.
\item A {\em thread} is an abstraction for a virtual program counter,
register set, execution stack and private data that is following some
control path (i.e. executing some program code) within an address space.
Threads are assigned to and always live within a context.  There
may be many threads assigned to a single context.
\end{itemize}
While the entire Ports group agrees on these terms, some feel that
this is not a complete model.  In particular, most massively parallel
machines support architectural features that make them more than
a set of nodes and a communication layer.  However, there was no
agreement on how to extend the definition set above without providing
undue complication in discussion of the topics that follow.

\item
{\em Communication.}  There are three important models of communication
that are used in current parallel programming paradigms.

\begin{itemize}

   \item Point-to-point communication between threads 
   in different contexts based on
   a message passing layer like MPI.

   \item Remote Service Requests (RSRs), where a message is sent to
   an address within a context in the same or another node  
    to invoke a function. The function is executed by a remote
   service request thread within that node.
   

   \item Hardware based "get" and "put" operations that allow
   a thread in one context to directly see or modify a "global address"
   which may lie in another context in the same or another node.

\end{itemize} 
Note that the last case is easily represented as a special, hardware
supported instance of a remote service request.  The Ports group could
not agree on whether it was better to focus on Point-to-point or
remote service request styles of communication.  However, it was
agreed that both were universal in that each could emulate the other.
It was also noted that most vendor supplied point-to-point systems were 
based on low level, specialized RSR mechanisms.  However,
it was also observed that a compiler could often generate point
to point communication patters that would contain less explicit
global synchronization  than a RSR scheme.  Of great
concern to the group was which scheme would provide the most
efficient communications. It was decided to wait until more experimental
data was made available.


\item
{\em Resources Management and Allocation.}  Being able to allocate
a new context and acquire new nodes is important for many applications.
Conversely, it is also important to be able to release resources.
Another important feature is to be able to connect to an existing
running computation.

\item
{\em Task Migration.} Despite the potential implementation
conflicts with heterogeneity, task/thread migration may be very
important for many applications.  In particular, load balancing
across nodes cannot be easily accomplished without this feature.
The discussion focused on the level of migration that would be 
allowed.  A running thread cannot migrate from one context to another
if it is referencing data that is not thread private: when moved,
the data (or even its address) may not exist the new context.
For example, if a thread allocates memory objects in the context heap
that is used for the lifetime of the thread, then the heap object
would need to be moved with the tread.
It was suggested that one possible mechanism would be to allow
threads that were ``pure'', in the sense that they made no reference
to context addresses other than thread local data, to be moved
if they were in some special state.  This is an area where
much more study is needed.

\item
{\em I/O } The ports group did not spend a long time on I/O
discussions but it was clear that much more remains to be
accomplished here.  In particular, there should be mechanism
for collective I/O operations from ropes of threads, atomicity
of I/O operations from individual threads and support for
parallel file systems.  This issue will be revisited in greater
detail in a future meeting.

\end{enumerate}

\section{Action Items}

There were three action items discussed.
\begin{itemize}
\item This Report would be circulated.

\item The Nexus and Chant groups would get together and discuss
the subset of pthreads that were considered essential to support.

\item Initial implementations of Nexus and Chant would be distributed
to interested parties as soon as possible.

\end{itemize}
The next meeting of the Ports group is scheduled for Aug. 16.

\end{document}



